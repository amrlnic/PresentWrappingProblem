\chapter{SAT}
The \textbf{Boolean Satisfiability} can be exploited in order to prove that the given ammount of presents, with the given dimensions can
fit in certain positions into the paper sheet. As far we have not numerical variables anymore we must reimplement from scratch the whole
models definition. We borrowed some concepts from the \textbf{CP} and \textbf{SMT} methods, but we had to port them into a new boolean logic.

\section{Base Model}
This model is the porting of the \textbf{SMT Base Model}, but we must describe the coordinates system with another variable.
Indeed, we loose all the variables of the precedent model, and we use a new tensor that will describe the whole problem. 

\begin{center}
    \begin{adjustwidth}{-1.5cm}{}
        \begin{tabular}{|c|c|c|}
            \hline
            \multicolumn{3}{|c|}{\textbf{Parameters}} \\
            \hline
            \textbf{Parameter} & \multicolumn{2}{|c|}{\textbf{Description}} \\
            \hline
            Width & \multicolumn{2}{|c|}{The Paper Sheet Width} \\
            \hline
            Height & \multicolumn{2}{|c|}{The Paper Sheet Height} \\
            \hline
            Presents & \multicolumn{2}{|c|}{The number of the Presents to place in the Paper Sheet} \\
            \hline
            Dimension X & \multicolumn{2}{|c|}{The array of the x dimensions of the Presents} \\
            \hline
            Dimension Y & \multicolumn{2}{|c|}{The array of the y dimensions of the Presents} \\
            \hline
            \multicolumn{3}{|c|}{\textbf{Extracted Parameters}} \\
            \hline
            \textbf{Parameter} & \textbf{Formula} & \textbf{Description} \\
            \hline
            Area & $Area = Width \cdot Height$ & Area of the Paper \\
            \hline
            Areas & $Areas[i] = Dimension_x[i] \cdot Dimension_y[i]$ & The array of the areas of the Presents \\
            \hline
            \multicolumn{3}{|c|}{\textbf{Variables}} \\
            \hline
            \textbf{Variable} & \multicolumn{2}{|c|}{\textbf{Description}} \\
            \hline
            Paper &  \multicolumn{2}{|c|}{A 3D boolean tensor describing the presence of the present in a particular position} \\
            \hline
        \end{tabular}
    \end{adjustwidth}
\end{center}

The $Paper$ tensor has two dimensions for indicating the present position and one dimension indicating the present index. In this way
we know that the i-th present will occupy the cell in the coordinates x, y if the boolean value of the $tensor[x, y, i]$ is true.

\subsection{Main Problem Constraints}
Now that the problem variables are decided, we can constraint the $Paper$ with some predicates, in \textbf{Propositional Logic}, in order
to carry out the solution of the problem.

\begin{itemize}
    \item[] \textbf{Essential Constraints}
    \item \textbf{\textit{Two different presents must not overlap:}}
    \begin{itemize}
        \item[] Given the two rectangles of two different presents, we can check if they have
            at least one part in common, just by checking if the tensor at position (x, y)
            holds in two different presents i and j. The \textit{overlaps} predicate is defined as:
        \item[] \begin{equation*}\begin{multlined}
            overlaps(Present_1, Present_2) \leftrightarrow \\
            \bigvee_{x, y \in Paper}(Paper[x, y, Present_1] \wedge Paper[x, y, Present_2])
        \end{multlined}\end{equation*}
    \end{itemize}
    \item \textbf{\textit{The presents must have and occupy the correct dimension:}}
    \begin{itemize}
        \item[] This was one of the hardes constrain to develop. We have to force the tensor to have the right
            ammount of true values in the correct place, for each present at a gien coordinate. The idea is
            that given a certain coordinate, we force the tensor to obbey a certain \textit{Disjunctive Normal Formula}.
        \item[] For each present, we fix a tuple of initial coordinates $(x_0, y_0)$ and we force the tensor to hold at 
                all the subsequent $Width \times Height$ coordinates, and not to hold the rest.
                Then we translate the initial coordinates and repeat the extraction of the formula.
                Once we have all the formulas for all the possible initial position of the present in the paper sheet,
                we concatenate them with an Or series into a \textit{Disjunctive Normal Formula}.
                Let's define the following predicate, where $p$ is the index of the current present:
        \item[] \begin{equation*}\begin{multlined}
            correct\_dimension(p, dx, dy) \leftrightarrow \\
            \bigvee_{
                \substack{
                    x_0 \in [1, Width - dx]\\
                    y_0 \in [1, Height - dy]
                }
            }
            (\bigwedge_{
                \substack{
                    x \in [x_0, x_0 + dx] \\
                    y \in [y_0, y_0 + dy]
                }
             } Paper[x, y, p])
             \vee
            (\bigwedge_{
                \substack{
                    x \in [1, x_0] \cup [x_0 + dx + 1, Width]\\
                    y \in [1, y_0] \cup [y_0 + dy + 1, Height]
                }
             } \neg Paper[x, y, p]) 
        \end{multlined}\end{equation*}
        \item[] So we end up with the full constrain:\\
        \begin{equation*} \bigwedge_{p \in [1, Presents]} correct\_dimension(p, Dimension_x[p], Dimension_y[p]) \end{equation*}  
    \end{itemize}
    \item \textbf{\textit{Each tensor tuple of coordinates must have at least one present:}}
    \item[] We want the tensor to have at least one present at each tuple of coordinates $(x, y)$:
    \begin{itemize}
        \item[] \begin{equation*}
            \bigwedge_{
                \substack{
                    x \in [1, Width]\\
                    y \in [1, Height]
                }
            } \bigvee_{p \in [1, Presents]} Paper[x, y, p]
        \end{equation*}
    \end{itemize} 
    \item[] \textbf{Additional Constraints}
    \item[] These constraint are not essential to solve the general formulation of this problem,
        but they results helpful as they restrict the search space in the given instances.
        The underlying assumption is that the instance contains the right amount of presents such
        that the area of the Paper Sheet is completely used.
    \item \textbf{\textit{}}
    \item \textbf{\textit{The presents must fill the row (column) dimension:}}
        \begin{itemize}
            \item[] We want to use each row \textit{(or column)} such that we use all of the available area of the paper.
            \item[] Drawing a vertical \textit{(horizontal)} we check that at least one present holds in the tensor in the line coordinates:
            \item[] Rows: \begin{equation*}\bigvee_{y \in [1, Height]} \bigwedge_{x \in [1, Width]} \bigvee_{p \in [1, Presents]} Paper[x, y, p]\end{equation*}
            \item[] Cols: \begin{equation*}\bigvee_{x \in [1, Width]} \bigwedge_{y \in [1, Height]} \bigvee_{p \in [1, Presents]} Paper[x, y, p]\end{equation*}
        \end{itemize}
\end{itemize}

\subsection{Results}
\begin{center}
    \begin{tabular}{|c|c|c|c|}
        \hline
        \multicolumn{4}{|c|}{\textbf{Results}} \\
        \hline
        \textbf{Instance} & \textbf{Time \textit{[s]}} & \textbf{Nodes} & \textbf{Propagations} \\
        \hline
    \end{tabular}
\end{center}

\section{Rotation Model}
As for \textbf{CP} and \textbf{SMT}, we just need another variable that keeps track of the rotation of each presnt in the paper sheet:

\begin{center}
    \begin{tabular}{|c|c|}
        \hline
        \multicolumn{2}{|c|}{\textbf{Variables}} \\
        \hline
        \textbf{Variable} & {\textbf{Description}} \\
        \hline
        Rotated & The boolean array that indicates whether a present is rotated or not \\
        \hline
    \end{tabular}
\end{center}

In this case, we do not need to use a proxy to gather the correct dimension, we just check the correct dimension in two different ways: the normal or the rotated one.
Like this, we can place each present in the normal \textit{OR} the rotated way and this is the resulting constrain: 
\begin{equation*}\begin{multlined}
    \bigwedge_{p \in [1, Presents]} (\\
        correct\_dimension(p, Dimension_x[p], Dimension_y[p]) \vee\\
        correct\_dimension(p, Dimension_y[p], Dimension_x[p]) \\
    )
\end{multlined}\end{equation*}  

As we can see, by switching the two dimension, we can simply rotate the present.


\begin{center}
    \begin{tabular}{|c|c|c|c|}
        \hline
        \multicolumn{4}{|c|}{\textbf{Results}} \\
        \hline
        \textbf{Instance} & \textbf{Time \textit{[s]}} & \textbf{Nodes} & \textbf{Propagations} \\
        \hline
    \end{tabular}
\end{center}

\section{Remarks and Results}
There are just a few of the implemented model because we wanted to devolop them just by using the Popositional Logic predicates, without recurring with Arithmetics
and Numerical calculus.\\

We briefly recap the overall results of the previous models in a textual informative table:

\begin{center}
    \begin{adjustwidth}{-1.5cm}{}
        \begin{tabular}{|c|c|c|c|c|}
            \hline
            \multicolumn{5}{|c|}{\textbf{Global Results}} \\
            \hline
            \textbf{Model} & \textbf{Speed} & \textbf{Complexity} & \textbf{Strengths} & \textbf{Weaknesses} \\
            \hline
        \end{tabular}
    \end{adjustwidth}
\end{center}